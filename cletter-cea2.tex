\documentclass[11pt]{letter}
\usepackage{hyperref}

% Enconding
\usepackage[utf8]{inputenc}
\usepackage[T1]{fontenc}

%\usepackage[portuguese]{babel}

\signature{Vitor Vasconcelos Araújo Silva}
%\address{Calle San Clemente 1, Piso 9, Puerta 4\\ Santa Cruz de Tenerife\\ 30008, SPAIN \\ vitors@cdtn.br \\ (+34) 634 807 860}
\address{Rua Conselheiro Dantas 6\\ Belo Horizonte, MG\\ 30411-305, BRAZIL\\ vitors@cdtn.br \\  (+34) 634 807 860}
% Acertar fechamento e assinatura para a esquerda
\longindentation=40pt

\begin{document}

%\begin{letter}{À l'équipe de recrutement du CEA}
\begin{letter}{HPC Monte-Carlo team}

  \opening{Dear Hiring manager,}
  
  I hold a full-time position since 2009 in the Nuclear Technology Development Centre, a public federal
  research institute, part of Brazilian Nuclear Energy Commission, where I work as a technology
  specialist and hold a permanent position. My job consists mainly on
  providing support to full-time researchers, in my case
  related to computer technologies - software development and infrastructure management - what I have been
  doing since 2011 in the group of Nuclear Reactors Technology, which is formed by no more than a dozen of researchers.
  Some of my activities consist in co-advising undergraduate and graduate students in computer related
  activities and Nuclear Engineering. I am also the administrator and responsible for the installation and management of a
  computer cluster of 9 nodes (with 180 cores). Despite not being part of the academics staff, on the first semester of 2019,
  I taught part of the course ``Computer Science applied to Thermal-Hydraulics'' as invited professor to the post-graduate 
  program of my Institute. A rewarding experience.

  Other of my activities consist in work on research/technical projects which, more recently, are the study of
  feasibility of GPU utilization by a Monte Carlo Nuclear code and the development of
  a simple visualization software for it and, recently, a study on the feasibility of digital neutron imaging on
  the TRIGA research reactor of my Institute. Since January 2020 I am on sabbatical unpaid leave initially planned to last one year extendable for two more.

  I got encouraged to apply to this position due to the activities expected to be performed. One of my professional
  passions is software development (parallel software development, more recently) and despite being able to keep doing that in the last 12 years, it has been only a complementary task among my activities. This means I only have used MPI, OpenMP and OpenCL/SYCL in simple problems in order to gain experience for further projects. I am aware that this can be a disadvantage on applying for this position. However, in my favor, I have experience on all the other duties expected for this position. To mention a few: C++ programming experience, knowledge in Python and Perl, vast experience in Unix-based OS and HPC. It is also worth mentioning my thesis work which involves the implementation of an algorithm for neutronics and thermal-hydraulics coupled calculations using POSIX shared memory API. It is worth mentioning that I worked as a full-time software developer (C++, Qt and CMake) before starting in my current position at INRIA Lorraine (Nancy, France).

%  I got encouraged to apply to this position due to the activities expected to be performed. These not only
%  match my experience but involve some of my main professional interests: high performance computing and
%  parallel software development. My attention was hooked by the opportunity to work with staff on the analysis
%  and optimizing software and workflows. %I know that software development is only a complementary task on the
  %duties of this position.
%  My recent experience on projecting, installing and managing a professional
%  cluster (a small one, but professional on the sense of equipment and tools employed, like a distributed file-system)
%  gave me a new perspective on the field of HPC infrastructure and I became really interested on it. However, at the sam

  %As a matter of fact, the activities listed on the job summary for this position are pretty much the activities I carry on a daily basis, but scaled to a whole new level, from a small scientific team to a whole university community. Certainly challenging but also exciting.
  
  At this point of my career, I think I am professionally mature enough to adapt and would like to apply my knowledge in a new challenging position: having a heavier part of my work directly related on software development applied to HPC, as aforementioned.

  There is also another important reason to venture into such change: being involved in different
  work culture in a different country - an experience I already had - and which I firmly believe is a driving force
  on getting better as a professional. I must also mention that the incertitudes in my country for the near future are another motivation in looking for a position where in a well known and respected research center where I can thrive as scientific software developer and researcher.

%  A detailed information on my technical skills is provided in my CV, but I should mention that besides meeting the minimum qualifications my skills also match the preferred qualifications, a huge part of my daily activities are done in front of a Linux terminal window writing shell scripts, managing a small HPC cluster and programming.
%  I believe it is worth mentioning I am entitled to ask for a non-payed leave lasting for a maximum of three years,
%  which I intend to use 
  %I am reaching a point professionaly where I have not much time left to try
  %new challenges.
%  It is common in my current institute to have people losing their enthusiasm due to the lack of
%  poor value given by the superior instances of the government to research activities in the nuclear field.
  Considering this, I believe I can both continue developing my professional skills while providing high quality work
  for the HPC Monte-Carlo group of CEA.
  
  \closing{Kind regards,}

%  \ps
%  P.S. I would be glad to be interviewed by Skype if it is of
%  your interest: \textbf{vitors.vasconcelos} is my contact on this
%  platform.
  
\end{letter}
\end{document}
