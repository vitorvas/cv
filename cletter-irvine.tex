\documentclass[11pt]{letter}
\usepackage{hyperref}

% Enconding
\usepackage[utf8]{inputenc}
\usepackage[T1]{fontenc}

%\usepackage[portuguese]{babel}

\signature{Vitor Vasconcelos Araújo Silva}
\address{Rua Tuiuti 4/402 \\ Belo Horizonte - MG \\ 30720-440, BRAZIL \\ vitors@cdtn.br \\ (+55) 31 99152 6349}

% Acertar fechamento e assinatura para a esquerda
\longindentation=40pt

\begin{document}

\begin{letter}{UCI Recruitment team}

  \opening{Dear Hiring Manager,}
  
  I hold a full-time tenure position since 2009 in the Nuclear Technology Development Centre, a public federal
  research institute, part of Brazilian Nuclear Energy Commission, where I work as a technology
  specialist and hold a position which can be compared to a tenure position in USA. My job consists mainly on
  providing support to full-time researchers, in my case
  related to computer technologies - software development and infrastructure management - what I have been
  doing since 2011 in the group of Nuclear Reactors Technology, which is formed by no more of a handful of researchers.
  Some of my activities consist in co-advising undergraduate and graduate students in computer related
  activities. I am also the administrator and responsible for the installation of a
  computer cluster of 9 nodes (with 180 cores).

  Other of my activities consist in collaboration
  to the open source project of a nuclear reactor
  analysis tool \textit{milonga}. I also work on research/technical projects which, nowadays, are the study of
  feasibility of GPU utilization by a Monte Carlo Nuclear code and the development of
  a simple visualization software for it.
  
  I got encouraged to apply to this position due to the activities expected to be performed. They not only
  match my experience but they involve some of my main professional interests: high performance computing
  and open source software utilization. %I know that software development is only a complementary task on the
  %duties of this position.
  My recent experience on projecting, installing and managing a professional
  cluster (a small one, but professional on the sense of equipment and tools employed) gave me a new perspective
  on the field of HPC infrastructure and I became really interested on the field.
  
  At this point of my career, I think I am professionally mature enough to apply my knowledge in a more
  challenging position. There is also another important reason to venture into such change: to work
  as part of a lively environment. Not to mention a different work culture in a different country - an
  experience which I already had - and which I firmly believe is a driving force on getting better as a professional.
%  I believe it is worth mentioning I am entitled to ask for a non-payed leave lasting for a maximum of three years,
%  which I intend to use 
  %I am reaching a point professionaly where I have not much time left to try
  %new challenges.
%  It is common in my current institute to have people losing their enthusiasm due to the lack of
%  poor value given by the superior instances of the government to research activities in the nuclear field.
  Considering this, I believe I do have much more to offer than I am being assigned in my current position. 
  It is tempting to rely on a tenure to work without pressure and comfortably, but I still want to do more and
  achieve more. UCI seems a place to do it.
  
  \closing{Kind regards,}

%  \ps
%  P.S. I would be glad to be interviewed by Skype if it is of
%  your interest: \textbf{vitors.vasconcelos} is my contact on this
%  platform.
  
\end{letter}
\end{document}
