\documentclass[11pt]{letter}
\usepackage{hyperref}

% Enconding
\usepackage[utf8]{inputenc}
\usepackage[T1]{fontenc}

\usepackage[french]{babel}

\signature{Vitor Vasconcelos Araújo Silva}
\address{Rua Tuiuti 4/402 \\ Belo Horizonte - MG \\ 30720-440, BRÉSIL \\ vitors@cdtn.br \\ (+55) 31 99152 6349}

% Acertar fechamento e assinatura para a esquerda
\longindentation=40pt

\begin{document}

\begin{letter}{À l'équipe de recrutement du CEA}

  \opening{Madame, Monsieur,}

  Je suis ingénieur informaticien et de développement, docteur en Sciences Nucléaires, titulaire du poste de technologiste chez Centre du Développement de la Technologie Nucléaire, un institut de recherche de la Commission Nationale d'Énergie Nucléaire du Brésil. Mes activités s'agitent, principalement, du développement logiciel appliqué à la thermohydraulique et la neutronique
  ensemble, de la gestion de l'infrastructure informatique au sein du group de travail sur la technologie des réacteurs
  nucléaires. Je m'occupe aussi de co-diriger les travails d'étudiants jusqu'au niveau
  doctorale et de la gestion de notre nouveaux système <<cluster>> avec 9 ordinateurs (180 coeurs).

  Autres des mes activités comprendre d'une collaboration au projet logiciel libre \textit{milonga}: outil d'analyse
  des réacteurs nucléaires. Je travail aussi sur des projets de recherche/techniques que sont, actuellement, des
  études de possibilité de utilisation de cartes graphiques (GPU) pour le code Monte Carlo Serpent2 et aussi sur un système
  de visualisation des géométries d'entrée des données pour le même code.

  Je suis encouragé a postuler au poste d'ingénieur de recherche en calcul scientifique principalement en raison de las activités attendues, que vont de rencontre a certaines de mes expériences mais aussi a certaines de mes intérêts professionnelles: le développement logiciel appliqué au domaine du nucléaire, spécialement de la méthode de Monte Carlo.

  À ce moment de mon parcours professionnel, je voudrais encore une fois expérimenter la culture de travail française
  et être dans une équipe de développement vif, ce que, malheureusement, je n'ai pas dans mon équipe actuelle, où je
  suis le seul développeur.

  Je dois mentionner que j'ai le droit de demander jusqu'à trois années congés non payes à mon employeur et j'en voudrais
  utiliser pour avancer professionnellement. Ce n'est pas un secret que la situation politique et social au Brésil a
  vraiment détérioré aux dernières années, ce que réfléchit directement sur les conditions de travail dans une organisme
  publique de recherche scientifique.

  %Je pense que m'éloigner de cette ambiance me ferá du bien professionalement et personalement.

  Compte tenu de cela, je crois que pour ceux que veulent évoluer techniquement au domaine du nucléaire, il n'y
  a pas meilleur endroit que le CEA.
  
  
  \closing{Mes salutations plus distinguées,}

%  \ps
%  P.S. Je suis prêt a être interviewé par skype si vous le voulez. Mon contát
%  skype est: \textbf{vitors.vasconcelos}.
  
\end{letter}
\end{document}
