\documentclass[11pt]{letter}
\usepackage{hyperref}

% Enconding
\usepackage[utf8]{inputenc}
\usepackage[T1]{fontenc}

\usepackage[french]{babel}

\signature{Vitor Vasconcelos Araújo Silva}
\address{Rua Tuiuti 4/402 \\ Belo Horizonte - MG \\ 30720-440, BRAZIL \\ vitors@cdtn.br \\ (+55) 31 99152 6349}

% Acertar fechamento e assinatura para a esquerda
\longindentation=40pt

\begin{document}

\begin{letter}{À l'équipe de recrutement du CEA}

  \opening{Madame, Monsieur,}

  Je suis ingénieur informatician et de développement, docteur en Sciences Nucleáires, titulaire du poste de techonologiste chez Centre du Développement da la Technologie Nucleáire, un institut de recherche de la Commission Nationale d'Energie Nucleaire. Mes activités s'agit, principalement, du développement logiciel apliqué à la thermohydraulique et la neutronique
  ensemble, de la gestion de l'infrastrucuture informatique au sein du group de travail sur la technologie des reacteurs
  nucleáires. Je m'occupe aussi de co-diriger les travails d'estudiants juscqu'au niveux
  doctorale et de la gestion de notre nouveaux système <<cluster>> avec 9 ordinateurs (180 coeurs).

  Autres des mes activités comprendre d'une collaboration au projet logiciel libre \textit{milonga}: outil d'analyse
  des reacteurs nucleáires. Je travail aussi sur des projets de recherche/techniques que sont, actuellement, des
  études de possibilité de utilisation de cartes graphiques (GPU) pour le code Monte Carlo Serpent2 et aussi sur un système
  de visualization des geometries d'entrée des données pour le même code.

  Je suis encouragé a postuler a ce poste là principalement en raison de las activités attendues, que vont de
  rencontre a certaines de mes expériences mais aussi a certaines de mes intérêts professioneles: le développement
  logiciel apliqué au domain du nucleáire, especialement de la méthod de Monte Carlo.

  À ce moment de mon parcour professionel, je voudrais encore une fois experiencer la culture de travail française
  et être dans une équipe de développement vif, ce que, malhereusement, je n'ai pas dans mon équipe actuélle, au je
  suis le seul développeur.

  Je dois mentioner que j'ai le droit de demander juscqu'à trois de congès non payes à mon employeur et j'en voudrais
  utiliser pour avancer professionalement. Ce n'est pas un secret que la situation politique et social au Brésil a
  vraiment détérioré aux derniéres années, ce que réfléchit directement sur les conditions de travail dans une organisme
  publique de rechèrche scientifique.

  %Je pense que m'éloigner de cette ambiance me ferá du bien professionalement et personalement.

  Compte tenu de cela, je crois que pour ceux que voulent évoluer techniquement au domain du nucleáire, il n'y
  a pas meilleur lieu que le CEA.
  
  
  \closing{Mes salutations plus distinguées,}

%  \ps
%  P.S. Je suis prêt a être interviewé par skype si vous le voulez. Mon contát
%  skype est: \textbf{vitors.vasconcelos}.
  
\end{letter}
\end{document}
